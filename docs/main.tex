% Incorporates https://gist.github.com/FelipeCortez/10729134

\documentclass[11pt,a4paper]{article}

% Package imports
\usepackage{titlesec}
\usepackage{geometry}
\usepackage{fancyhdr}
\usepackage{graphicx}
\usepackage{hyperref} % \url, https://www.overleaf.com/learn/latex/Hyperlinks
\usepackage{outlines} % better itemize
\usepackage{comment}
\usepackage{multirow} % tables
\usepackage{noto-sans} % Google Noto Fonts
\usepackage{hyperref} % Hyperlinks to sectiosn
\usepackage{listings}
\usepackage{lstautogobble} % listings: Fix relative indenting
\usepackage{color} % listings: Code coloring
\usepackage{zi4} % listings: Nice font

\usepackage[british]{datetime2} % load before gitinfo2 to customize
\usepackage[mark=true,grumpy=true]{gitinfo2}

\usepackage{subfiles} % Best loaded last in the preamble

% Redefine \gitMark to customize it
% https://mirror.apps.cam.ac.uk/pub/tex-archive/macros/latex/contrib/gitinfo2/gitinfo2.pdf
\renewcommand{\gitMark}{Branch: \gitBranch\,@\,\gitAbbrevHash{}\,\textbullet{}\,\DTMusedate{gitdate}}

% Use sans font (noto sans) as default font for this document
\renewcommand{\familydefault}{\sfdefault}

% Word style normal margins.
\geometry{a4paper, includeheadfoot, portrait, total={}, top=12.5mm, bottom=12.5mm, left=25.4mm, right=25.4mm}

\graphicspath{ {./images/} }

% Variables
\def\projectname{Inventory Project}

% Override subparagraph with a variant that has no indentation
% https://tex.stackexchange.com/a/392014
\makeatletter
\renewcommand\subparagraph{%
\@startsection{subparagraph}{5}{0pt}%
{3.25ex \@plus 1ex \@minus .2ex}{-1em}%
{\normalfont\normalsize\bfseries}}
\makeatother

\title{\projectname}
\author{James Cahill}
\date{Sepetember 2023}

% Configure fancyHDR page style
% https://tex.stackexchange.com/questions/266911/get-fancyhdr-and-geometry-to-work-nicely
\fancypagestyle{style}{
    \fancyhead{} % clear all header fields
    \fancyhead[HL]{\projectname}
    \fancyhead[HR]{James Cahill}
    \renewcommand{\headrulewidth}{0pt} % Remove header line
}
\pagestyle{style}

% Configure listings colours
\definecolor{bluekeywords}{rgb}{0.13, 0.13, 1}
\definecolor{greencomments}{rgb}{0, 0.5, 0}
\definecolor{redstrings}{rgb}{0.9, 0, 0}
\definecolor{graynumbers}{rgb}{0.5, 0.5, 0.5}

% Configure listings style
\lstset{
    autogobble,
    columns=fullflexible,
    showspaces=false,
    showtabs=false,
    breaklines=true,
    showstringspaces=false,
    breakatwhitespace=true,
    escapeinside={(*@}{@*)},
    commentstyle=\color{greencomments},
    keywordstyle=\color{bluekeywords},
    stringstyle=\color{redstrings},
    numberstyle=\color{graynumbers},
    basicstyle=\ttfamily\footnotesize,
    frame=l,
    framesep=12pt,
    xleftmargin=12pt,
    tabsize=4,
    captionpos=b
}


\begin{document}


\tableofcontents

\pagebreak

\section{Analysis}

% Background
\subsection{Problem Identification}

\subsubsection{Problem Description}
\subfile{sections/analysis/problemIdentification/problemDescription}

\subsubsection{Stakeholders}
\subfile{sections/analysis/problemIdentification/stakeholders}

\subsubsection{Why is it suitable to a computational solution?}

\begin{comment}
why creating this solution is better with the use of technology
eg:
need a way to store large amounts of data; perfect for a database
easy way to add/remove inventory (would be labour intensive otherwise - paper based systems)
can be v. easily done with a gui

identifying key things the solution should have; explain why doing
this computationally is a good idea / is easy

1/2 a page to a page

eg decomposition/abstraction
decomp:
large program; by splitting into smaller sub-programs
can make each one individually and combine at the end
explain how they can be used to achieve the goals/impls
\end{comment}

\subsection{Investigation}

\subsubsection{Preparation for interview}
\subfile{sections/analysis/investigation/prep}

\subsubsection{Interviews}

\begin{comment}
2 or 3

person
question
answer
brief summary
\end{comment}

\subsubsection{Summary of interviews}

\begin{comment}
half a page of key things that were found out from the interviews
should include / should not include / etc.


\end{comment}

\pagebreak

\subsection{Research}


\subsubsection{Existing similar solutions}
\subfile{sections/analysis/research/existing}

\subsubsection{Features to be incorporated into solution}

\begin{comment}
based on research etc
select the features from the research that will be incorporated
and explain what they do
from sortly, steal x feature because y
include things you won't include as well (out of scope), because xyz
\end{comment}

\subsubsection{Limitations of the solution}

\begin{comment}
limitations:

- time

- limited by any software

- money - hosting backend?
- not getting an apple dev account
so won't be a "true" mobile app, more of a website on the home screen.
\end{comment}

\subsubsection{Feedback from stakeholders}

\subsection{Requirements}

\subsubsection{Stakeholder requirements}
\label{sec:stakeholderRequirements}

\pagebreak

\subsubsection{Software and hardware requirements}
\subfile{sections/analysis/requirements/software_hardware}

\pagebreak

\subsubsection{Success requirements}
\subfile{sections/analysis/requirements/success}

\section{Design}

\begin{comment}
design: for each page/screen:

picture of page


brief desc of what the page will do

for each one show the stakeholder requirements or success requirements that will be met when this page/feature is implmented;


then break down each component of the design page.
sentance or two on what it does and why (justify it being there)

\end{comment}

\subsection{User Interface Design}

\subsubsection{Usability Features}

\subsubsection{Feedback from stakeholder}

\subsection{Modular breakdown}

\subsection{Algorithms}

\subsection{Data Dictionary}

\subsection{Inputs and outputs}

\subsection{Validation}

\subsection{Testing}

\subsubsection{Methods}

\subsubsection{Test Plan}

\pagebreak

\section{Implementation}

\subsection{First Iteration | Initial Backend and Database}
\subfile{sections/implementation/first_iteration.tex}


\pagebreak

\section{Testing}

\section{Evaluation}

\end{document}