% Incorporates https://gist.github.com/FelipeCortez/10729134

\documentclass{article}

% Package imports
\usepackage{titlesec}
\usepackage{geometry}
\usepackage{fancyhdr}
\usepackage{graphicx}
\usepackage{hyperref} % \url, https://www.overleaf.com/learn/latex/Hyperlinks
\usepackage{outlines} % better itemize
\usepackage{comment}
\usepackage{multirow} % tables
\usepackage{noto} % Google Noto Fonts
\usepackage{hyperref} % Hyperlinks to sectiosn
\usepackage{listings}
\usepackage{lstautogobble} % listings: Fix relative indenting
\usepackage{color} % listings: Code coloring
\usepackage{zi4} % listings: Nice font


\usepackage[british]{datetime2} % load before gitinfo2 to customize
\usepackage[mark=true,grumpy=true]{gitinfo2}

\usepackage{subfiles} % Best loaded last in the preamble


% Redefine \gitMark to customize it
% https://mirror.apps.cam.ac.uk/pub/tex-archive/macros/latex/contrib/gitinfo2/gitinfo2.pdf
\renewcommand{\gitMark}{Branch: \gitBranch\,@\,\gitAbbrevHash{}\,\textbullet{}\,\DTMusedate{gitdate}}

% Word style normal margins.
\geometry{a4paper, includeheadfoot, portrait, total={}, top=12.5mm, bottom=12.5mm, left=25.4mm, right=25.4mm}

\graphicspath{ {./images/} }

% Variables
\def\projectname{Inventory Project}

% Override subparagraph with a variant that has no indentation
% https://tex.stackexchange.com/a/392014
\makeatletter
\renewcommand\subparagraph{%
\@startsection{subparagraph}{5}{0pt}%
{3.25ex \@plus 1ex \@minus .2ex}{-1em}%
{\normalfont\normalsize\bfseries}}
\makeatother

\title{\projectname}
\author{James Cahill}
\date{Sepetember 2023}

% Configure fancyHDR page style
% https://tex.stackexchange.com/questions/266911/get-fancyhdr-and-geometry-to-work-nicely
\fancypagestyle{style}{
    \fancyhead{} % clear all header fields
    \fancyhead[HL]{\projectname}
    \fancyhead[HR]{James Cahill}
    \renewcommand{\headrulewidth}{0pt} % Remove header line
}
\pagestyle{style}

% Configure listings colours
\definecolor{bluekeywords}{rgb}{0.13, 0.13, 1}
\definecolor{greencomments}{rgb}{0, 0.5, 0}
\definecolor{redstrings}{rgb}{0.9, 0, 0}
\definecolor{graynumbers}{rgb}{0.5, 0.5, 0.5}

% Configure listings style
\lstset{
    autogobble,
    columns=fullflexible,
    showspaces=false,
    showtabs=false,
    breaklines=true,
    showstringspaces=false,
    breakatwhitespace=true,
    escapeinside={(*@}{@*)},
    commentstyle=\color{greencomments},
    keywordstyle=\color{bluekeywords},
    stringstyle=\color{redstrings},
    numberstyle=\color{graynumbers},
    basicstyle=\ttfamily\footnotesize,
    frame=l,
    framesep=12pt,
    xleftmargin=12pt,
    tabsize=4,
    captionpos=b
}


\begin{document}


\tableofcontents

\pagebreak

\section{Analysis}

% Background
\subsection{Problem Identification}

\subsubsection{Problem Description}
\subfile{sections/analysis/problemIdentification/problemDescription}

\subsubsection{Stakeholders}
\subfile{sections/analysis/problemIdentification/stakeholders}

\subsubsection{Why is it suitable to a computational solution?}

\begin{comment}
why creating this solution is better with the use of technology
eg:
need a way to store large amounts of data; perfect for a database
easy way to add/remove inventory (would be labour intensive otherwise - paper based systems)
can be v. easily done with a gui

identifying key things the solution should have; explain why doing
this computationally is a good idea / is easy

1/2 a page to a page

eg decomposition/abstraction
decomp:
large program; by splitting into smaller sub-programs
can make each one individually and combine at the end
explain how they can be used to achieve the goals/impls
\end{comment}

\subsection{Investigation}

\subsubsection{Preparation for interview}
\subfile{sections/analysis/investigation/prep}

\subsubsection{Interviews}

\begin{comment}
2 or 3

person
question
answer
brief summary
\end{comment}

\subsubsection{Summary of interviews}

\begin{comment}
half a page of key things that were found out from the interviews
should include / should not include / etc.


\end{comment}

\pagebreak

\subsection{Research}


\subsubsection{Existing similar solutions}
\subfile{sections/analysis/research/existing}

\subsubsection{Features to be incorporated into solution}

\begin{comment}
based on research etc
select the features from the research that will be incorporated
and explain what they do
from sortly, steal x feature because y
include things you won't include as well (out of scope), because xyz
\end{comment}

\subsubsection{Limitations of the solution}

\begin{comment}
limitations:

- time

- limited by any software

- money - hosting backend?
- not getting an apple dev account
so won't be a "true" mobile app, more of a website on the home screen.
\end{comment}

\subsubsection{Feedback from stakeholders}

\subsection{Requirements}

\subsubsection{Stakeholder requirements}
\label{sec:stakeholderRequirements}

\pagebreak

\subsubsection{Software and hardware requirements}
\subfile{sections/analysis/requirements/software_hardware}

\pagebreak

\subsubsection{Success requirements}
\subfile{sections/analysis/requirements/success}

\section{Design}

\begin{comment}
design: for each page/screen:

picture of page


brief desc of what the page will do

for each one show the stakeholder requirements or success requirements that will be met when this page/feature is implmented;


then break down each component of the design page.
sentance or two on what it does and why (justify it being there)

\end{comment}

\subsection{User Interface Design}

\subsubsection{Usability Features}

\subsubsection{Feedback from stakeholder}

\subsection{Modular breakdown}

\subsection{Algorithms}

\subsection{Data Dictionary}

\subsection{Inputs and outputs}

\subsection{Validation}

\subsection{Testing}

\subsubsection{Methods}

\subsubsection{Test Plan}

\pagebreak

\section{Implementation}

\subsection{First Iteration | Initial Backend and Database}

\begin{comment}
\subsubsection{Introduction OLD}

\noindent\\ The first thing I decided to work on was user account creation. This would involve using GraphQL to receive data about the new user, generating the user ID, and then placing that data in a database.
After doing some research (should I talk about the different options?) I decided to use the following three libraries for my project:

\begin{outline}
    \1 \textbf{Gin}\\
    Gin is a HTTP framework for Go. This allows me to expose my GraphQL endpoint over HTTP. After conducting some research, I choose to use gin as it is by far the fastest library available at the time of writing.
    It offers performance up to 40 times faster than it's closest alternative, \textit{"Martini"}.
    Gin is a very popular framework, with over 72 thousand stars on GitHub. Using this library benefits me as it has been battle-tested by other users and includes end to end testing, making it more than suitable for my project. (wording isn't great, trying to say it has been tested already)

    \1 \textbf{Gqlgen}\\
    Gqlgen is a Go library designed to make building GraphQL servers easy and hassle-free. It is designed with a schema-first approach, meaning the developer (me) can simply define their API using the standard GraphQL Schema Definition Language. It prioritizes type safety, and has rudimentary validation support built-in. However, I will not be relying on this validation feature, I will create my own validation system that better suits my needs.

    \1 \textbf{GORM} (Go Object Relational Mapping library)\\
    GORM is a fully-features ORM library for Go. (explain what an ORM does)
\end{outline}

\end{comment}

\subsubsection{Introduction}

In this sprint I will work on the backend service.
This service will provide an interface for the frontend to talk to the database via an API (Application Programming Interface).
I am writing the backend in Go, which is a performant, statically typed high level language designed by Google.
Go is frequently used for backend development thanks to it's excellent performance and built in memory safety.
I am also going to use \underline{GraphQL} as the query language used by the frontend to interact with the backend.

\noindent \\ GraphQL is an open-source query and manipulation language designed for use in APIs. The backend will serve as an API which will interface with my database.
I choose to use GraphQL as it is better suited for larger, more complex data sources, and supports querying for multiple different types of data at once, unlike REST. It is also something I was interested in learning more about as I have not designed a system using it before. \\

James: Should this intro be in the design area instead?

TODO: explain what a graphql mutation is (it's a function)

\subsubsection{User account creation}

The first feature I decided to work on was user account creation. This would involve asking the user for an email address, name and password, before validating it and inserting it into the database.
In addition, at a later stage, validation must be performed in order to ensure that:

\begin{itemize}
    \item The user email is not already in use
    \item The generated user ID is unique and not already in use
\end{itemize}

\noindent For this early stage of development, I decided to use an SQLite database to make things easier. I can easily switch this to PostgreSQL (as specified in my design doc) later.

\noindent \\ When a GraphQL mutation is executed to create a new user, the GraphQL library calls the \lstinline{CreateUser} function, passing any input data from the query.\\ The database connection is available at "\lstinline{r.db}".\\

\noindent My first version of this function was as follows:

\begin{lstlisting}[language=Go]
// CreateUser is the resolver for the createUser field.
func (r *mutationResolver) CreateUser(ctx context.Context, input model.NewUser) (*model.User, error) {
	// Create the user struct
    user := structs.User{FirstName: input.FirstName, LastName: input.LastName, Email: input.Email}
	
    // Generate a user ID
    user.ID = uuid.New()

    // Create the database entry
	r.db.Create(&user)

    // Return the created user data, converting it to a GraphQL model.
	return &model.User{
        ID: user.ID.String(),
        FirstName: user.FirstName,
        LastName: user.LastName,
        Email: user.Email
    }, nil // return nil in the error field
\end{lstlisting}

\noindent \\ The first thing I noticed after implementing this function was that it was tedious to convert back and forth between \lstinline{structs.User} (the database object) and \lstinline{model.User} (the GraphQL object). I decided to merge these into a single object. This was done with the following lines in my GraphQL library's configuration file:

\begin{lstlisting}[language=Python]
models:

[..]

    # Custom models
    Checkout:
      model: github.com/jcxldn/fosscat/backend/structs.Checkout
    Entity:
      model: github.com/jcxldn/fosscat/backend/structs.Entity
    Item:
      model: github.com/jcxldn/fosscat/backend/structs.Item
    User:
      model: github.com/jcxldn/fosscat/backend/structs.User
\end{lstlisting}

This instructs the GraphQL library to use the structs I defined for the database as if they were GraphQL models.

\subparagraph{Unique ID generation}

\noindent \\\\ I decided to use UUIDs (Universal Unique Identifiers) as IDs for all of the objects in my database. (eg. Users, Items) As seen above, I initially choose to simply call \lstinline{uuid.New()} to generate a new random UUID. However, I would soon realise that it would be beneficial to perform \textbf{validation} in order to ensure that the UUIDs were actually unique, ie. that they were not being used by another object of the same type. For example, I wouldn't want two users to have the same User ID.

\noindent \\ I decided to use a \textbf{for loop} to continuously generate UUIDs to be used as a potential User ID. I then perform a database lookup to ensure that the UUID is not already in use.

This can be done with the following code:

\begin{lstlisting}[language=Go]
func (r *mutationResolver) CreateUser(ctx context.Context, input model.NewUser) (*structs.User, error) {

[..]

	isFreeUuid := false
	for !isFreeUuid {
		// Generate a UUID for the user id.
		user.ID = uuid.New()
		// Check that the UUID has not been used already
		// If true, it will break out of this for loop and continue.
		isFreeUuid = util.IsUuidFree[structs.User](r.db, user.ID)
	}

\end{lstlisting}

\noindent In order to achieve this and reduce code duplication across different functions, \\
I created a "IsUuidFree" utility function. Here is the code:

\begin{lstlisting}[language=Go]

func IsUuidFree[T any](db *gorm.DB, id uuid.UUID) bool {
	obj := new(T)
	err := db.Model(obj).Select("id").Where("id = ?", id.String()).First(&obj).Error
	if errors.Is(err, gorm.ErrRecordNotFound) {
		// Record not found, so user id is free
		return true
	} else {
		// Record was returned successfully, therefore the user exists
		return false
	}
}

\end{lstlisting}

JAMES: this initially was different but I changed it to make it simpler. Should I include the old version as well?\\

\noindent This function makes use of \textbf{generics}. As per the Go docs:

\begin{quotation}
    \textit{
        With generics, you can declare and use functions or types that are written to work with any of a set of types provided by calling code.
    }
\end{quotation}

\noindent To simplify, generics mean that I can pass any struct (\textbf{T}) to the function.
For example, if I call the function with:\\

\begin{lstlisting}[language=Go]
    util.IsUuidFree[structs.User](r.db, user.ID)
\end{lstlisting}

\noindent \\Then T is set to the type \lstinline{structs.User}.\\

\noindent \textbf{GORM} (my database library) works by defining a struct to query for which corresponds to a table in the database (in this case the \lstinline{Users} struct corresponds to the \lstinline{users} table). We can then perform SQL actions on this table, such as Select. \\

\noindent Therefore, the GORM db call from above:

\begin{lstlisting}[language=Go]
    db.Model(obj).Select("id").Where("id = ?", id.String()).First(&obj)
\end{lstlisting}

\noindent is the equivalent of:

\begin{lstlisting}[language=SQL]
    SELECT id FROM users WHERE id == ? LIMIT 1 VALUES ("value of id.String()")
\end{lstlisting}

\section{Testing}

\section{Evaluation}

\end{document}