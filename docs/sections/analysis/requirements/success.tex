\documentclass[../../../main.tex]{subfiles}

\begin{document}

The overall objectives for the system.

\noindent To measure the overall effectiveness of the system, targets must be set before writing the program.
These targets will help in the evaluation stage to determine weather our objectives have been met.
These objectives will be \textbf{SMART}, i.e:

\begin{outline}
    \1 \textbf{Specific}\\
    What objective needs to be accomplished?
    \1 \textbf{Measurable}\\
    How can we quantify this objective?\\
    How will the success of this objective be measured? (quantitatively or qualitatively)
    \1 \textbf{Achievable}\\
    Is this objective achievable and realistic? If so, how to you plan to achieve them?
    \1 \textbf{Relevant}\\
    How does this objective benefit the end-users of this application as a whole?\\
    Why has this goal been set?
    \1 \textbf{Timely}\\
    Can this objective be completed within an appropriate time frame?\\
    At what stage in the software development lifecycle will you start implementing this goal?\\
    In which order will any sub-objectives be completed?
\end{outline}

\paragraph{The Project's SMART Objectives}

\begin{enumerate}
    \item \textbf{To produce a solution for cataloguing a school library and recording users and books borrowed}\\
          At the end of the project, I will evaluate against my success criteria and determine weather this objective has been met.
          On the software side, I will be using React, Expo and PostgreSQL. This objective will be the main objective for this project.
          This objective must be completed by \underline{March 2024}.

    \item \textbf{To produce a solution including a database that can store details of books, borrowers, loans and returns}\\

    \item \textbf{To produce an intuitive and easy to use solution}\\
          I will evaluate my success on this objective by having a new user without any prior training or advice use the system and
          try to carry out a number of tasks without any assistance. If the user is able to successfully complete the tasks
          I will consider the system to be intuitive and easy to use and therefore this objective satisfied.
          To achieve this I will design my system to have a consistent layout based on \textbf{Material Design 2}, (\url{https://m2.material.io/})
          the design language used by Google products and many apps running on the Android operating system. I will also use language that is
          a) appropriate for the situation the product will be deployed in (with young children) and b) easy to understand (so that children can interact with the system)
          I will also use meaningful error messages so that the user has a clear understanding of the problem that has occurred.

    \item \textbf{To produce a solution that features a fully searchable catalogue}

    \item \textbf{To produce a solution that features reporting for overdue and/or lost books}
    \item \textbf{To produce a solution that includes a curated "suggested reading list" for each borrower}
\end{enumerate}

\begin{comment}
GS:
Produce a system that manages inventory in a statistical and written/informative format
having the information automatically produce a graph
to clearly show when stock is low

to have a supplementary phone app that can be downloaded with the provision to
scan and check in/out items, stored in a database.

quantiative manner (numbers wise)
can update a spreadsheet automatically

able to be used remotely (T in SMART)

data inputted to the system will be processed within a time period to produce a usable outcome
(techy stuff goes here)

make use of qr scanning library to easily scan QR codes that will be placed on

display specific information about different items depending on their type
- number of inventory available
- highlight stock that is low

predict how much you are spending on consumables
eg. budgeting as an objective us ea library

asset value figure for budgeting
are you within the budget or not
\end{comment}

\end{document}